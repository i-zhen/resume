\documentclass[11pt,a4paper,sans]{moderncv}

\usepackage{amsmath}
\usepackage{fontspec} %important
\usepackage{fontawesome}
% moderncv themes
\moderncvstyle{banking}
\moderncvcolor{blue}

%%%%%%%%%%%%%%%%%%%%%%%%%%%%%%%%%%
\renewcommand*{\namefont}{\fontsize{50}{52}\mdseries\upshape}
%%%%%%%%%%%%%%%%%%%%%%%%%%%%%%%%%%

%\usepackage[top=3.2cm, bottom=3.2cm, left=3.2cm, right=3.2cm]{geometry}
\usepackage[scale=0.78]{geometry}
%\usepackage[scale=0.9]{geometry}

\definecolor{blue}{rgb}{0.22, 0.45, 0.698}
\newcommand\Colorhref[3][blue]{\href{#2}{\small\color{#1}#3}}

\usepackage{xeCJK}
%\setsansfont{Monaco}

\setCJKmainfont{楷体-简}
\setCJKsansfont{黑体-简}
\setCJKmonofont{楷体-简}

% personal data
\name{甄}{羿}
\title{Curriculum Vitae}
\address{上海市浦东新区浦三路817弄}{11-502}{}
\mobile{+86~177~1768~7328}
\email{iamzhenyi@gmail.com}
\homepage{izhen.me}
\social[github]{i-zhen} 
\social[linkedin]{izhenyi}   
%\extrainfo{additional information}   
%\photo[64pt][0.4pt]{picture}        
%\quote{聪明在于学习,天才在于积累}     

\begin{document}
\makecvtitle
%\maketitle

\section{工作经历}
%\cventry{02/2014--04/2014}{研究助理}{广东省计算科学重点实验室(中山大学)}{}{}{配置软件,代码调试与调优,文档撰写}
\begin{itemize}
	\item 2018.11 -- 至今, 软件工程师, \Colorhref[blue]{https://www.huawei.com/cn/corporate-information}{\textit{华为技术有限公司}}, 中国上海
	\item 2017.04 -- 2018.11, 软件工程师, \Colorhref[blue]{http://www.citigroup.com/china/csts/}{\textit{花期金融服务(中国)有限公司}}, 中国上海
\end{itemize}

\section{项目经验}
\subsection{工作项目}
\cventry{2019.11 -- 至今}{双向预览、低代码相关工作}{\textsf{低代码平台}}{}{}{
	\begin{itemize}
		\item 作为tech lead负责双向预览项目的下游依赖拉通对齐,涉及:渲染引擎,运行时与语言框架
		\item 作为tech lead负责领导架构低代码平台,实现codegen,元模型与DSL
		\item 类SwiftUI功能开发,实现了基于BFS与最长不降子序列的动态规划diff算法模块
		\item *HMS转换工具:帮助HMS开发者快速从GMS完成API切换
	\end{itemize}
}

\cventry{2018.11 -- 2019.10}{Dex2Maple编译器}{\textsf{方舟编译器}}{}{}{
	\begin{itemize}
		\item 研究与实现invoke-custom指令的前端静态化(细节机密)
		\item 实现类smali的方舟前端lexer和parser(未开源,机密)
	\end{itemize}
}

\cventry{2017.08 -- 2018.11}{大宗交易代理的转账系统。客户可以使用系统预订转账和收入}{\textsf{WIRE}}{}{}{
	\begin{itemize}
		\item 负责货币风险模块的全栈开发,前端借助Ext JS运用MVC模式实现了CRUD与审批功能以及风险参考值的数据钻取(下钻分析)。后端运用界面、工厂、动态代理等设计模式实现了控制层、服务层、DAO层
		\item \small 为下游数据仓库团队开发对账模块:通过有限状态机将业务逻辑抽象为类似TCP三次握手的实时核对/重发协议。实现了异步反序列化消息的功能
		\item \small 通过分析日志,调试并修复现存bug。解决客户的紧急需求、修复Production Issue、持续集成
	\end{itemize}
}

\cventry{2017.04 -- 2017.08}{包含数据采集与风险值分析和数据管理功能的反洗钱风控平台}{\textsf{EWARA}}{}{}{
	\begin{itemize}
		\item \small 实现复杂树形条目的审批业务逻辑。使用JDBC开发可以用于多张表的通用DAO层
		\item \small 辅助参与邮件提醒功能的编写、RESTful API的设计、PDF报表功能的维护
		\item \small 使用Mockito技术编写单元测试,应用反射技术扩展POJO测试模块,结合H2数据库编写集成测试
	\end{itemize}
}

\subsection{个人项目}
\cventry{2016.11 -- 2017.01}{web 应用}{\textsf{Lambda计划}}{创始人}{}{
	使用Haskell实现的类 Hacker News 社会化消息分享平台,偏向IT业界和计算机科学的内容。用户可以发布新闻、学术内容以及一般提问。减少时间浪费于无营养信息是本平台的主要目标
	\begin{description}
		\item[\textsf{关键字}] Haskell 8.0.2, Scotty, Persistent, mime-mail, websockets, Blaze, PostgreSQL, Bootstrap, jQuery
		\item[Github \textsf{地址}] \texttt{https://github.com/ProLambda/Times}
		\item[\textsf{相关博文地址}] \texttt{https://izhen.me/2017/08/20/aws-lambda/}
	\end{description}
}

%\item{
\cventry{2016.06 -- 2016.08}{Haskell 库 - 3106次下载(2021.01.17)}{\textsf{PPrinter: A generic derivable Haskell pretty printer}}{作者}{}{
	PPrinter 是一个为用户自定义任意类型自动推导漂亮打印(pretty printing)函数的Haskell库(GHC 编译器的派生机制支持自动生成 Typeclass 实例的函数)
	\begin{description}
		\item[\textsf{关键字}] Dissertation Project, Hackage, Haskell 7.10.2
		\item[Hackage \textsf{地址}] \texttt{http://hackage.haskell.org/package/PPrinter-0.1.0}
	\end{description}
}
%}

\cventry{2015.10 -- 2015.12}{系统软件}{\textsf{Small-C语言编译器}}{开发者}{}{
	为C程序设计语言的子集(未包含指针,宏等内容)使用Java语言实现了一个简单且没有优化的编译器。拥有词法分析,语法分析,语义分析和代码生成四个基本模块
	\begin{description}
		\item[\textsf{关键字}] Java 7, ASM 4
		\item[Github \textsf{地址}] \texttt{https://github.com/i-zhen/Reactor-C}
	\end{description}
}
\cventry{2015.10 -- 2015.10}{本地应用}{\textsf{类ML语言的解释器}}{开发者}{}{
	使用Scala程序设计语言实现的一款简单的类ML一阶函数式编程语言的解释器,支持语法糖,递归函数和lambda算子
	\begin{description}
		\item[\textsf{关键字}] Scala 2.11.7
		\item[Github \textsf{地址}] \texttt{https://github.com/i-zhen/Apache-Longbow}
	\end{description}
}

\subsection{校队经验}
\cventry{2014}{参赛队员}{\textsf{中山大学 ASC(世界大学生超级计算机竞赛) 校集训队}}{}{}{
    \begin{itemize}
	\item ASC14 需要5名队员相互协作,在运行功率不超过3000W的限制下将HPC性能发挥到最大
        	\item 学习了基本的数值计算方法,串行算法并行化,接触并了解了异构计算与多处理器程序设计方法
	\item 主要职责是在决赛阶段主要负责优化SU$^{2}$单项 -- 斯坦福大学开发的PDE 流体力学 C++ 代码
	\end{itemize}
}


\section{专业技能}
\cvitem{语言}{Haskell, Java, C/C++, TypeScript, Python, Scala, Rust, Prolog, Coq, Isabelle}
\cvitem{框架}{Spring, JMS, JUnit, Mockito, JDBC, Java servlet, SQL}

\section{竞赛荣誉}
\cventry{2014}{ASC世界大学生超级计算机竞赛 2014 (ASC14)}{\textsf{一等奖(Linpack测试破世界纪录)}}{}{}{HPL测试破世界纪录,并赢得奖金人民币10000元}
\cventry{2014}{ACM-ICPC广东省大学生程序设计竞赛 2014}{\textsf{铜牌}}{}{}{}
\cventry{2009 \& 2008}{全国青少年信息学奥林匹克联赛(NOIP) 2009 \& 2008}{\textsf{两次获得一等奖}}{}{}{}


\section{教育经历}
\cventry{2015.09 -- 2016.11}
{人工智能 - 理学硕士}
{\textsf{爱丁堡大学}}
{英国爱丁堡市}{} 
{毕业论文: Deriving Pretty-printing for Haskell, 指导教授: Prof. Philip Wadler}

\cventry{2011.09 -- 2015.06}
{软件工程 - 工学学士}
{\textsf{中山大学}}
{中国广东省广州市}{} 
{全国青少年信息学奥林匹克联赛(高中组)一等奖,保送至中山大学}


\end{document}
