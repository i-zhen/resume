\documentclass[11pt,a4paper,sans]{moderncv}

\usepackage{amsmath}

% moderncv themes
\moderncvstyle{banking}
\moderncvcolor{blue}  

\usepackage[utf8]{inputenc}
\usepackage[scale=0.785]{geometry}

\definecolor{blue}{rgb}{0.22, 0.45, 0.698}
\newcommand\Colorhref[3][blue]{\href{#2}{\small\color{#1}#3}}

\usepackage{import}

% personal data
\name{Yi}{Zhen}
\title{Curriculum Vitae}
\address{888, Xinzha Rd, Jingan District}{Shanghai 200000}{China}
\mobile{+86~177~1768~7328}
\email{iamzhenyi@gmail.com}
\homepage{izhen.me}
\social[github]{i-zhen} 
\social[linkedin]{izhenyi} 
%\quote{Keep It Simple and Stupid} 

\begin{document}
\makecvtitle

\section{Experience}
\begin{itemize}
	\item Mar. 2021 -- Present, Senior Software Engineer(P7), \Colorhref[blue]{https://www.antgroup.com}{\textit{Ant Group}}, Shanghai, P.R.China
	\item Nov. 2018 -- Mar. 2021, Software Engineer, \Colorhref[blue]{https://www.huawei.com/en/gallery/facilities/hw_u_202617}{\textit{Huawei Research}}, Shanghai, P.R.China
	\item Apr. 2017 -- Nov. 2018, Software Engineer, \Colorhref[blue]{http://www.citigroup.com/china/csts/}{\textit{Citigroup}}, Shanghai, P.R.China
\end{itemize}

\section{Job Projects}
%\item{
\cventry{Mar. 2021 -- Present}{Research and development, project manager of Gödel project}{\textsf{Project Manager; R\&D Engineer}}{Ant Group}{}{
	\begin{description}
	    \item[Gödel Data Analytics Engine] Datalog-based Query Programming Language; Powered by Soufflé
		\item[Semi-Parsing System] Persistent Robust(Fault Tolerance) Compiling System for C/C++/ObjC
	\end{description}
}
%\}

%\item{
\cventry{Nov. 2018 -- Mar. 2021}{Development and leading the architecture design}{\textsf{Tech Lead; R\&D Engineer}}{Huawei Research}{}{
	\begin{description}
	    \item[Bitfun Codeless Platform] Lowcode Platform for UI Design; SwiftUI-like Plugin for VSCode/IntelliJ
		\item[HMS Toolkit] A plug-in for integrating the HMS Core for developers
		\item[Ark Compiler] Compiling Android Dex to MapleVM/Runtime
	\end{description}
}
%\}

%\item{
\cventry{Apr. 2017 -- Nov. 2018}{Development and Continuous Integration}{\textsf{Development Engineer}}{Citigroup}{}{
	\begin{description}
	    \item[WIRE] Wire Transfer system for block trading agent
		\item[EWARA] Anti money laundering platform
		\item[HAMSTER] AI project for bank statement understanding
	\end{description}
}
%}

\section{Miscellaneous}
\subsection{Side Projects}
%\begin{itemize}
%\item{
\cventry{Jun. 2022 -- Jul. 2022}{web3 app}{\textsf{ERC721G: Anti-Theft ERC721 Solution}}{author}{}{
World's First Anti-Theft NFT Series and Solution: A universal technical solution to be adapted to any ERC721 standard smart contract to help NFT holders guard their property security jointly
	\begin{description}
		\item[Github Address] \texttt{https://github.com/turtlecasedao/OpenERC721G}
		\item[OpenSea Address] \texttt{https://opensea.io/collection/turtle-case-gang}
	\end{description}
}
%}

%\item{
\cventry{Nov. 2016 -- Jan. 2017}{web2 app}{\textsf{Project Lambda}}{founder}{}{
   	A Hacker-News-like social information platform focusing on IT industry and computer science, which users could publish general news, academic contents and questions through it
	\begin{description}
		\item[Keywords] Haskell 8.0.2, Scotty, Persistent, mime-mail, websockets, Blaze, PostgreSQL, Bootstrap, jQuery
		\item[Github Address] \texttt{https://github.com/ProLambda/Times}
		\item[Chinese Blog Article] \texttt{https://izhen.me/2017/08/20/aws-lambda/}
		%\item[Hello] ~
		%	\begin{itemize}
		%		\item Machine learning and its applications in natural language processing
		%	\end{itemize}
	\end{description}
}
%}

%\item{
\cventry{Jun. 2016 -- Aug. 2016}{Haskell Library - 1413 times download till 05/27/2018}{\textsf{PPrinter: A generic derivable Haskell pretty printer}}{author}{}{
   	PPrinter is a Haskell library that supports automatic derivation of pretty printing functions on user defined arbitrary data types (the deriving mechanism supports the automatic generation of instances for functions)
	\begin{description}
		\item[Keywords] Dissertation Project, Hackage, Haskell 7.10.2
		\item[Hackage Address] \texttt{http://hackage.haskell.org/package/PPrinter-0.1.0}
	\end{description}
}
%}

%\item{
\cventry{Oct. 2015 -- Dec. 2015}{system software}{\textsf{Compiler of Small-C}}{developer}{}{
   	A compiler for the subset of C language that compiles the source code to Java bytecode. It contains the essential parts of a standard compiler including lexer, parser, semantic analyzer and code generator
		\begin{description}
		\item[Keywords] Java 7, ASM 4
		\item[Github Address] \texttt{https://github.com/i-zhen/Reactor-C}
	\end{description}
}
%}

%\item{
\cventry{Oct. 2015 -- Oct. 2015}{local app}{\textsf{Interpreter of ML-like Programming Language}}{developer}{}{
   	An interpreter written in scala for a simple ML-like programming language which supports syntactic sugar, type checking, recursive function and first-order lambda calculus
	\begin{description}
		\item[Keywords] Scala 2.11.7
		\item[Github Address] \texttt{https://github.com/i-zhen/Apache-Longbow}
	\end{description}
}
%}

\subsection{School Team}
\cventry{2014}{Team member}{Sun Yat-sen University ASC Student Supercomputer Challenge Team}{}{}{
    \begin{itemize}
    	\item ASC14 required the team to wring the most HPC performance out of a 3000W power allowance
        	\item Mastered the numerical methods,  relevant algorithms, heterogeneous and multiprocessor programming
	\item Optimized the SU$^{2}$ -- a Stanford University developed open-source C++ code for PDE analysis and designed things that adhere to PDE constraints, and assisted in the HPL event
	\end{itemize}
}

%\end{itemize}

\section{Professional Skills}
\subsection{Programming Language and Framework}
\cvitem{Language}{Rust, Solidity, TypeScript, Haskell, C/C++, Java, Python, Scala, Prolog, Coq, SQL}
\cvitem{Framework}{Hardhat, web3.js/ether.js, clangd/vscode-clangd, Spring, JMS, Mockito, JDBC}

\section{Awards}
\cventry{2014}{\Colorhref[blue]{https://www.asc-events.org/ASC14/index14en.php}{\textit{The ASC Student Supercomputer Challenge (ASC14)}}}{First Prize and Highest Linpack Award}{}{}{Set a new world record of HPL(Linpack) performance and won 10,000 CNY}
\cventry{2014}{\Colorhref[blue]{https://icpc.baylor.edu/regionals/finder/provincial-nine-one-2014}{\textit{The ACM-ICPC China Guangdong Provincial Programming Contest (GDCPC)}}}{Bronze Medal}{}{}{}
\cventry{2009\&2008}{\Colorhref[blue]{http://www.noi.cn/newsview.html?id=107&hash=6C159E&type=8}{\textit{The National Olympiad in Informatics in Provinces (NOIP)}}}{Two-time recipient of First Prize}{}{}{}

\section{Education}
\cventry{Sep. 2015 -- Nov. 2016}
{Master of Science in Artificial Intelligence}
{University of Edinburgh}
{Edinburgh, U.K.}{}
{Dissertation: Deriving Pretty-printing for Haskell, supervised by Prof. Philip Wadler}

\cventry{Sep. 2011 -- Jun. 2015}
{Bachelor of Engineering in Software Engineering}
{Sun Yat-sen University}
{Guangzhou, P.R.China}{}
{Recommended for admission to SYSU and exempted from Gaokao because of well performance at NOIP}

\end{document}
