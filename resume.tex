\documentclass[11pt,a4paper,sans]{moderncv}

\usepackage{amsmath}

% moderncv themes
\moderncvstyle{banking}
\moderncvcolor{blue}  

\usepackage[utf8]{inputenc}
\usepackage[scale=0.78]{geometry}

\definecolor{blue}{rgb}{0.22, 0.45, 0.698}
\newcommand\Colorhref[3][blue]{\href{#2}{\small\color{#1}#3}}

\usepackage{import}

% personal data
\name{Yi}{Zhen}
\title{Curriculum Vitae}
\address{97-401, Jinxiu Rd, Pudongxin District}{Shanghai 200000}{China}
\mobile{+86~177~1768~7328}
\email{iamzhenyi@gmail.com}
\homepage{izhen.me}
\social[github]{i-zhen} 
\social[linkedin]{izhenyi} 
%\quote{Keep It Simple and Stupid} 

\begin{document}
\makecvtitle

\section{Experience}
\begin{itemize}
	\item Apr. 2017 -- Present, Software Engineer, \Colorhref[blue]{http://www.citigroup.com/china/csts/}{\textit{Citigroup Inc.}}, Shanghai, P.R.China
	\item Feb. 2014 -- Apr. 2014, Research Assistant Intern, \Colorhref[blue]{http://www.compsci.sysu.edu.cn/cn/cn01/index.htm}{\textit{Guangdong Province Key Laboratory of Computational Science}}, Guangzhou, P.R.China
\end{itemize}

\section{Education}
\cventry{Sep. 2015 -- Nov. 2016}
{Master of Science in Artificial Intelligence}
{University of Edinburgh}
{Edinburgh, U.K.}
{\textit{Pass with Merit}}
{Dissertation: Deriving Pretty-printing for Haskell, supervised by Prof. Philip Wadler}

\cventry{Sep. 2011 -- Jun. 2015}
{Bachelor of Engineering in Software Engineering}
{Sun Yat-sen University}
{Guangzhou, P.R.China}
{GPA \textit{: 3.3}}
{Recommended for admission to SYSU and exempted from Gaokao because of well performance at NOIP}

\subsection{Major Courses}
{\small Machine Learning \& Pattern Recognition, Automated Reasoning,  Secure Programming,  Types and Semantics for Programming Languages, Compiler Optimisation, Artificial Intelligence, Algorithm Design and Analysis, Data Mining, Numerical Computation Methods}

\section{Skills}
\subsection{Programming Language and Framework}
\cvitem{Languages}{Java, C/C++, Python, Haskell, Scala, Prolog, Coq, Scheme}
\cvitem{Frameworks \& Tools}{Spring, Ext JS, JUnit, Mockito, JDBC, JMS, Java servlet, SQL, TensorFlow, Tomcat, Git, SonarQube, Spark, Amazon EC2, NuSMV, Isabelle, Vim, \LaTeX}
\subsection{Professional Skills}
\begin{itemize}
	\item Machine Learning and Knowledge Representation (Knowledge Graph)
	\item Full-stack development, RESTful API design and Continuous Integration for web application
\end{itemize}

\section{Miscellaneous}
\subsection{School Team}
\cventry{2014}{Team member}{Sun Yat-sen University ASC Student Supercomputer Challenge Team}{}{}{
    \begin{itemize}
    	\item ASC14 required the team to wring the most HPC performance out of a 3000W power allowance
        	\item Mastered the numerical methods,  relevant algorithms, heterogeneous and multiprocessor programming
	\item Optimized the SU$^{2}$ -- a Stanford University developed open-source C++ code for PDE analysis and designed things that adhere to PDE constraints, and assisted in the HPL event
	\end{itemize}
}

\subsection{Projects}
%\begin{itemize}
%\item{
\cventry{Nov. 2016 -- Jan. 2017}{web app}{\textsf{Project Lambda}}{founder}{}{
   	A Hacker-News-like social information platform focusing on IT industry and computer science, which users could publish general news, academic contents and questions through it. Reducing time wasting on nonnutritive information is the major goal
	\begin{description}
		\item[Keywords] Haskell 8.0.2, Scotty, Persistent, mime-mail, websockets, Blaze, PostgreSQL, Bootstrap, jQuery
		\item[Github Address] \texttt{https://github.com/ProLambda/Times}
		%\item[Hello] ~
		%	\begin{itemize}
		%		\item Machine learning and its applications in natural language processing
		%	\end{itemize}
	\end{description}
}
%}

%\item{
\cventry{Jun. 2016 -- Aug. 2016}{package}{\textsf{PPrinter: A generic derivable Haskell pretty printer}}{author}{}{
   	PPrinter is a Haskell library that supports automatic derivation of pretty printing functions on user defined arbitrary data types (the deriving mechanism supports the automatic generation of instances for functions)
	\begin{description}
		\item[Keywords] Dissertation Project, Hackage, Haskell 7.10.2
		\item[Hackage Address] \texttt{http://hackage.haskell.org/package/PPrinter-0.1.0}
	\end{description}
}
%}


%\item{
\cventry{Oct. 2015 -- Dec. 2015}{system software}{\textsf{Compiler of Small-C}}{developer}{}{
   	A compiler for the subset of C language that compiles the source code to Java bytecode. It contains the essential parts of a standard compiler including lexer, parser, semantic analyzer and code generator
		\begin{description}
		\item[Keywords] Java 7, ASM 4
		\item[Github Address] \texttt{https://github.com/i-zhen/Reactor-C}
	\end{description}
}
%}

%\item{
\cventry{Oct. 2015 -- Oct. 2015}{local app}{\textsf{Interpreter of ML-like Programming Language}}{developer}{}{
   	An interpreter written in scala for a simple ML-like programming language which supports syntactic sugar, type checking, recursive function and first-order lambda calculus
	\begin{description}
		\item[Keywords] Scala 2.11.7
		\item[Github Address] \texttt{https://github.com/i-zhen/Apache-Longbow}
	\end{description}
}
%}

%\item{
\cventry{Apr. 2013 -- Jun. 2013}{web app}{\textsf{The Student Activity Center(SAC) Room Reservation System}}{full-stack developer}{not open source}{
   	Designed a room reservation system for the Student Activity Center in Sun Yat-sen University. Students can use this system to book rooms in SAC and managed their own information
	\begin{description}
		\item[Keywords] Python 2.7.5, Javascript, MySQL, web.py, Bootstrap, jQuery
	\end{description}
}
%}

%\item{
\cventry{Mar. 2012 -- Apr. 2012}{server-side module}{\textsf{The Kernel of Online Judge}}{indie developer}{}{
   	This is an online system that can compile and execute codes and then test them with pre-constructed data. Users submit code and run it with restrictions (time limit, memory limit, etc.). The output of the code will be compared with the standard output and then return the result to the users
	\begin{description}
		\item[Keywords] C/C++, Linux, Linux-API
		\item[Github Address] \texttt{https://github.com/i-zhen/Simple-OJ-core}
	\end{description}
}
%}

%\end{itemize}

\section{Awards}
\cventry{2014}{The ASC Student Supercomputer Challenge (ASC14)}{First Prize and Highest Linpack Award}{}{}{Set a new world record of HPL(Linpack) performance and won �10,000 CNY}
\cventry{2014}{The ACM-ICPC China Guangdong Provincial Programming Contest (GDCPC)}{Bronze Medal (Third Prize)}{}{}{}
\cventry{2009\&2008}{The National Olympiad in Informatics in Provinces (NOIP)}{Two-time recipient of First Prize}{}{}{}

\end{document}
